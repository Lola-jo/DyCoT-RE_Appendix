\documentclass{article}  % 使用单栏文档类
\usepackage[utf8]{inputenc}
\usepackage[
    left=2.5cm,    % 左边距
    right=2.5cm,   % 右边距
    top=2cm,       % 上边距减小到2cm
    bottom=2cm,    % 下边距减小到2cm
    includeheadfoot % 包含页眉页脚在内
]{geometry}
\usepackage{cite}
\usepackage{amsmath,amssymb,amsfonts}
\usepackage{algorithmic}
\usepackage{graphicx}
\usepackage{textcomp}
\usepackage{xcolor}
\usepackage[linesnumbered,ruled,vlined]{algorithm2e}
\usepackage{hyperref}
\usepackage{booktabs}
\usepackage{makecell}
\usepackage{multirow}
\usepackage{algorithm}
\usepackage{algorithmic}
\usepackage{listings}
\usepackage{color}
\usepackage{longtable}
\definecolor{dkgreen}{rgb}{0,0.6,0}
\definecolor{gray}{rgb}{0.5,0.5,0.5}
\definecolor{mauve}{rgb}{0.58,0,0.82}
\lstset{
    frame=tb,
    language=Python,
    aboveskip=3mm,
    belowskip=3mm,
    showstringspaces=false,
    columns=fixed,
    basicstyle={\footnotesize\ttfamily},
    numbers=left,
    numberstyle=\tiny\color{gray},
    keywordstyle=\color{blue},
    commentstyle=\color{dkgreen},
    stringstyle=\color{mauve},
    breaklines=true,
    breakatwhitespace=true,
    escapeinside=``,
    tabsize=4,
    extendedchars=false,
    captionpos=t,
    linewidth=\textwidth,     % 使用全部文本宽度
    xleftmargin=1em,          % 减小左边距
    xrightmargin=1em,         % 减小右边距
    resetmargins=true
}

\DeclareRobustCommand*{\IEEEauthorrefmark}[1]{%
    \raisebox{0pt}[0pt][0pt]{\textsuperscript{\footnotesize\ensuremath{#1}}}}

\def\BibTeX{{\rm B\kern-.05em{\sc i\kern-.025em b}\kern-.08em
    T\kern-.1667em\lower.7ex\hbox{E}\kern-.125emX}}

\setlength{\LTleft}{0pt}     % 左边距设为0
\setlength{\LTright}{0pt}    % 右边距设为0

\begin{document}


\appendix

\section{Algorithm}
    \indent The reward engineering process with Chain-of-Thought (CoT) reasoning is detailed in Algorithm 1. This algorithm presents a systematic approach to iteratively improve reward functions through a combination of task-oriented initialization and reflection-guided refinement.

\begin{algorithm}[H]
\caption{Reward Engineering with Chain-of-Thought}
\label{alg:reward_engineering}
\begin{algorithmic}[1]
\STATE Initialize environment and best\_reward = (code='', fitness=$-\infty$)

\STATE // Iteration 0: Task-oriented CoT
\STATE prompt = BuildInitialPrompt(task\_description, env\_code)
\IF{has\_initial\_reward}
    \STATE prompt += InitialRewardSuggestion()
\ENDIF
\STATE reward\_codes = GenerateWithCoT(prompt, n\_samples)
\STATE best\_reward = EvaluateRewards(reward\_codes)

\STATE // Iterations 1+: Reflection-guided CoT
\FOR{iteration = 1 \TO max\_iterations}
    \STATE components = AnalyzeComponents(best\_reward)
    \STATE reflection = CreateReflection(components)
    \STATE prompt = BuildReflectivePrompt(reflection, components)
    
    \STATE // CoT-guided reward generation
    \STATE reward\_codes = GenerateWithCoT(prompt, n\_samples)
    \FOR{each code in reward\_codes}
        \STATE fitness = TrainAndEvaluate(code)
        \IF{fitness > best\_reward.fitness}
            \STATE best\_reward = (code, fitness)
        \ENDIF
    \ENDFOR
\ENDFOR

\RETURN best\_reward

\STATE // Chain of Thought generation process
\STATE analysis = "1. Analysis of previous performance"
\STATE reasoning = "2. Improvement strategy"
\STATE implementation = "3. Reward function code"
\RETURN LLM.generate(analysis + reasoning + implementation)
\end{algorithmic}
\end{algorithm}

\section{Experimental Details}

\subsection{Task Details}
    \indent The BipedalWalker-v3 environment from Gymnasium presents a challenging control task where a bipedal robot must learn to walk efficiently across terrain. The agent receives observations including hull angle, angular velocity, horizontal speed, vertical speed, position of joints and joints angular speed, legs contact with ground, and 10 lidar rangefinder measurements. The action space consists of 4 continuous actions controlling the torques applied to the joints. The episode terminates if the robot reaches the goal, falls over, or exceeds the maximum number of steps.

\subsection{Implementation Details}
    \indent The implementation configuration details are summarized in Table 1, which shows the key parameters used across different components of the system.

\begin{table}[htbp]
    \caption{Implementation Configuration Parameters}
    \label{tab:config}
    \begin{center}
    \begin{tabular}{|l|l|l|}
    \hline
    \textbf{Component} & \textbf{Parameter} & \textbf{Value} \\
    \hline
    \multirow{7}{*}{Main} 
    & Backend & ollama \\
    & Model & llama3.1 \\
    & Temperature & 0.7 \\
    & Iterations & 8 \\
    & Samples & 8 \\
    & Sleep Time & 5 min \\
    & Use Initial Reward & false \\
    \hline
    \multirow{4}{*}{CoT} 
    & Enabled & true \\
    & Performance Weight & 0.7 \\
    & Thought Process Weight & 0.3 \\
    & Max Tokens & 2000 \\
    \hline
    \multirow{4}{*}{Environment} 
    & Name & bipedal\_walker \\
    & Class Name & BipedalWalker \\
    & Max Episode Steps & 1600 \\
    & Benchmark & BipedalWalker-v3 \\
    \hline
    \multirow{11}{*}{RL Algorithm} 
    & Type & PPO \\
    & Policy & MlpPolicy \\
    & Network Architecture & [64, 64] \\
    & Learning Rate & 0.0003 \\
    & Batch Size & 64 \\
    & N Steps & 2048 \\
    & N Epochs & 10 \\
    & Gamma & 0.999 \\
    & GAE Lambda & 0.95 \\
    & Clip Range & 0.2 \\
    & Max Grad Norm & 0.5 \\
    \hline
    \multirow{6}{*}{Training} 
    & Total Timesteps & 1,000,000 \\
    & Number of Envs & 10 \\
    & Device & auto \\
    & State Stack & 1 \\
    & Eval Episodes & 8 \\
    & Eval Seeds & [5, 10] \\
    \hline
    \multirow{3}{*}{Logging} 
    & Log Directory & logs/bipedal\_walker \\
    & Log Level & INFO \\
    & Save GIF & true \\
    \hline
    \end{tabular}
    \end{center}
\end{table}

The configuration spans multiple components: the main system settings, Chain of Thought parameters, environment specifications, reinforcement learning algorithm parameters, and training settings. As shown in Table 1, we used the llama3.1 model with a temperature of 0.7 for generation, and implemented PPO as the base RL algorithm with carefully tuned hyperparameters. The CoT component was enabled with a balanced weighting between performance (0.7) and thought process evaluation (0.3).

\section{Prompt Details}
    \indent The prompting system consists of three main stages for reward function generation. Each stage employs specific prompts to guide the generation process. 

\subsection{Stage 1: Base Prompts and Task Context}
    \indent The first stage combines foundational system prompts with task-specific context. Table 2 and Table 3 present the initial system prompt and coding instructions that form the foundation of our approach. These base prompts establish the core requirements and constraints for reward function generation.
    \indent Following the base prompts, we define the specific task requirements and environment context. For the BipedalWalker task, this includes the task description (Table 4), environment implementation details (Table 5), and an optional initial reward template (Table 6). These components provide crucial details for the reward function design and establish the baseline implementation context.

\begin{table*}[htbp]
    \caption{Base System Prompts}
    \begin{center}
    \begin{tabular}{|p{0.98\textwidth}|}
    \hline
    \textbf{Initial System Prompt} \\
    You are a reward engineer writing effective reward functions for reinforcement learning tasks. \\
    Your goal is to create a reward function to help the agent learn the task described in text. \\
    Use relevant environment variables as inputs. Example signature: \\
    \texttt{def compute\_reward(self, ...):} \\
    \texttt{\quad return reward, \{\}} \\
    Do not use type hints. Return type: float, Dict[str, float]. \\
    \hline 
    \end{tabular}
    \end{center}
    \label{tab:base_prompts}
    \end{table*}

\begin{table*}[htbp]
\caption{Coding Instructions}
\label{tab:coding_instructions}
\begin{center}
\begin{tabular}{|p{0.98\textwidth}|}
\hline
\textbf{Coding Instructions} \\
The reward function output should include: \\
1. Total reward (float) \\
2. Dictionary of individual reward components \\
\\
Format as a Python code string. Tips: \\
1. Normalize rewards using transformations like np.exp \\
2. Ensure input types match expected types \\
3. Use only self attributes from environment class definition \\
4. Initialize self variables with null checks \\
5. No new input variables \\
6. Include proper code formatting \\
7. Pass self as first argument \\
8. Do not compute fitness\_score components \\
9. Create only the reward function \\
10. Maintain smooth transitions \\
\hline
\end{tabular}
\end{center}
\end{table*}

\begin{table*}[htbp]
\caption{Task-Specific Components}
\begin{center}
\begin{tabular}{|p{0.98\textwidth}|}
\hline
\textbf{Task Description (BipedalWalker)} \\
The goal is to make a biped navigate a 2D environment: \\
- The biped starts standing at the left end of the terrain with the hull horizontal \\
- The biped has to reach the right end and avoid falling down \\
- Must keep moving forward, avoid getting stuck \\
- Optimize for speed and minimal episode steps \\
- Goal achievement prioritized over movement smoothness \\
\hline
\end{tabular}
\end{center}
\label{tab:task_components}
\end{table*}

The implementation details of the environment are provided in Table 5, which shows the core step function of the BipedalWalker environment. This function handles the physics simulation and state updates, providing the foundation for our reward engineering process.

Based on the environment implementation, we optionally develop an initial reward template as shown in Table 6. When used, this template serves as a starting point for our CoT-based reward function enhancement process, incorporating basic components such as progress tracking, energy efficiency, and stability measures.

The environment implementation details are shown in Table 5, which presents the core step function of the BipedalWalker environment.

\begin{longtable}{|p{\textwidth}|}
\caption{Step Function Implementation of BipedalWalker}
\label{tab:step_implementation} \\

\hline
\textbf{Step Function Implementation} \\
\endfirsthead

\multicolumn{1}{c}{\tablename\ \thetable{} -- continued from previous page} \\
\hline
\textbf{Step Function Implementation (continued)} \\
\endhead

\hline
\multicolumn{1}{r}{\textit{Continued on next page}} \\
\endfoot

\hline
\endlastfoot

\begin{lstlisting}[language=Python, frame=none, aboveskip=0pt, belowskip=0pt, xleftmargin=1em]
    def step(self, action: np.ndarray):
        for i in range(4):
            self.joints[i].motorSpeed = float(SPEED_HIP * np.sign(action[i]))
            self.joints[i].maxMotorTorque = float(
                MOTORS_TORQUE * np.clip(np.abs(action[i]), 0, 1)
            )
    
        self.world.Step(1.0 / FPS, 6 * 30, 2 * 30)
    
        pos = self.hull.position
        vel = self.hull.linearVelocity
    
        for i in range(10):
            self.lidar[i].fraction = 1.0
            self.lidar[i].p1 = pos
            self.lidar[i].p2 = (
                pos[0] + math.sin(1.5 * i / 10.0) * LIDAR_RANGE,
                pos[1] - math.cos(1.5 * i / 10.0) * LIDAR_RANGE,
            )
            self.world.RayCast(self.lidar[i], self.lidar[i].p1, self.lidar[i].p2)
    
        state = [
            self.hull.angle,
            2.0 * self.hull.angularVelocity / FPS,
            0.3 * vel.x * (VIEWPORT_W / SCALE) / FPS,
            0.3 * vel.y * (VIEWPORT_H / SCALE) / FPS,
        ]
        state += [l.fraction for l in self.lidar]
    
        terminated = self.game_over or pos[0] < 0 or \
                    pos[0] > (TERRAIN_LENGTH - TERRAIN_GRASS) * TERRAIN_STEP
    
        reward, individual_reward = self.compute_reward(pos, action, state, terminated)
        fitness_score = self.compute_fitness_score(pos, action, state, terminated)
        individual_reward.update({'fitness_score': fitness_score})
    
        return np.array(state, dtype=np.float32), reward, terminated, False, individual_reward
    \end{lstlisting}
\end{longtable}


The reward function implementation generated by our approach is presented in Table 6, which can be used as a starting point for the reward function generation process.


\begin{longtable}{|p{\textwidth}|}
\caption{Reward Function Implementation for BipedalWalker}
\label{tab:reward_implementation} \\

\hline
\textbf{Reward Function Implementation} \\
\endfirsthead

\multicolumn{1}{c}{\tablename\ \thetable{} -- continued from previous page} \\
\hline
\textbf{Reward Function Implementation (continued)} \\
\hline
\endhead

\hline
\multicolumn{1}{r}{\textit{Continued on next page}} \\
\endfoot

\hline
\endlastfoot

\begin{lstlisting}[language=Python, frame=none]
def compute_reward(self, pos, action, state, terminated):
    # Distance-based reward: reduced scale and increased precision
    distance_reward = np.exp(pos.x / 2000.0) - 1

    # Stability reward:
    angle_penalty = np.tanh(np.abs(state[0]) / 0.01) * 0.05
    angular_velocity_penalty = np.tanh(np.abs(state[1]) / 0.01) * 0.05

    # Smoothness reward:
    action_penalty = np.tanh(-np.sum(np.abs(action)) / len(action) / 10.0) * -0.2

    # Obstacle avoidance penalty: reduced magnitude
    obstacle_penalty = -np.min([l.fraction for l in self.lidar]) * 1

    # Velocity penalty: increased magnitude
    velocity_penalty = np.tanh(np.abs(state[2]) / 0.5) * -1.5

    # Jump penalty: reduced magnitude
    jump_penalty = np.tanh(-state[3] / 2.0) * 0.5

    # Proximity to goal reward:
    proximity_to_goal_reward = -np.exp(-(pos.x - TERRAIN_LENGTH) ** 2 / 50.0) * 1.5

    # Total reward: adjusted weights and scales for improved balance
    reward = (
        distance_reward
        + 0.3 * angle_penalty
        + 0.4 * angular_velocity_penalty
        - 0.6 * action_penalty
        - 0.1 * obstacle_penalty
        - 0.8 * velocity_penalty
        - 0.25 * jump_penalty
        + 2.0 * proximity_to_goal_reward
    )

    # Termination conditions remain the same

    individual_reward = {
        "distance_reward": distance_reward,
        "angle_penalty": angle_penalty,
        "angular_velocity_penalty": angular_velocity_penalty,
        "action_penalty": action_penalty,
        "obstacle_penalty": obstacle_penalty,
        "velocity_penalty": velocity_penalty,
        "jump_penalty": jump_penalty,
        "proximity_to_goal_reward": proximity_to_goal_reward,
    }

    return reward, individual_reward

\end{lstlisting}
\end{longtable}

\subsection{Stage 2: Initial CoT Generation (Iteration 0)}
    \indent The second stage focuses on the initial Chain-of-Thought (CoT) generation process. Table 7 presents the detailed prompt template used in the first iteration (Iteration 0) of our reward engineering process. This systematic prompt guides the language model through a careful analysis of the task requirements and constraints before implementing the actual reward function.


\begin{table*}[htbp]
\caption{Initial CoT Generation Prompt (Iteration 0)}
\begin{center}
\begin{tabular}{|p{\textwidth}|}
\hline
\begin{verbatim}
Before implementing the reward function, please follow these steps:

1. First, analyze the task description and environment code carefully:
   - What is the main goal?
   - What are the key state variables?
   - What are the constraints?

2. Identify the key objectives and constraints:
   - Primary objectives
   - Secondary objectives
   - Physical constraints
   - Action constraints

3. Break down how different actions should be rewarded:
   - What behaviors should be encouraged?
   - What behaviors should be discouraged?
   - How to balance exploration vs exploitation?

4. Consider potential edge cases:
   - Boundary conditions
   - Extreme scenarios
   - Potential exploitation of the reward

5. Design your reward components:
   - Base reward components
   - Shaping rewards
   - Normalization approach

Format your response as follows:
1. Start with "Reasoning:" followed by your detailed analysis of each step above
2. Then provide your code implementation in a Python code block (```python ... ```)
\end{verbatim} \\
\hline
\end{tabular}
\end{center}
\label{tab:cot_generation}
\end{table*}


The initial Chain-of-Thought (CoT) generation process follows a structured approach to reward function design. Table 7 presents the detailed prompt template used in the first iteration (Iteration 0) of our reward engineering process. This systematic prompt guides the language model through a careful analysis of the task requirements and constraints before implementing the actual reward function.

\subsection*{Stage 3: Iterative Refinement (Iterations 1 to n)}
The final stage implements an iterative refinement process using three key components shown in Tables 8, 9, and 10:

1. Performance Analysis Prompt (Table 8): Evaluates the previous iteration's performance through training metrics, fitness scores, and behavioral patterns. This prompt helps identify strengths and weaknesses in the current reward function.

2. CoT Reflection Prompt (Table 9): Builds upon the performance analysis to guide structured thinking about potential improvements. This prompt maintains consistency with the initial CoT approach while incorporating learned insights.

3. Reflection Instructions Prompt (Table 10): Provides specific guidelines for analyzing reward components and their effectiveness. This prompt focuses on practical aspects such as scaling, component relevance, and fitness score optimization.

This iterative process continues until the desired performance is achieved or the maximum number of iterations is reached. Each iteration builds upon the insights gained from previous rounds, incorporating feedback and performance metrics to guide the optimization process.


\begin{table*}[htbp]
    \caption{Performance Analysis Prompt}
    \label{tab:performance_analysis}
    \begin{center}
    \begin{tabular}{|p{\textwidth}|}
    \hline
    \begin{verbatim}
    Previous Iteration Performance Analysis:

    1. Training Metrics:
    {training_metrics}
    
    2. Current Best Reward Function:
    {best_reward_code}
    
    3. Performance Statistics:
    - Maximum achieved fitness: {max_fitness}
    - Average episode length: {avg_episode_length}
    - Learning stability: {learning_stability}
    - Success rate: {success_rate}
    
    4. Key Observations:
    - Learning curve characteristics
    - Agent behavior patterns
    - Critical events or milestones
    - Potential issues or limitations
    
    Please analyze this information using the Chain of Thought process to design 
    an improved reward function.
    \end{verbatim} \\
    \hline
    \end{tabular}
    \end{center}
    \end{table*}

    \begin{table*}[htbp]
        \caption{CoT Reflection Prompt}
        \label{tab:cot_reflection}
        \begin{center}
        \begin{tabular}{|p{\textwidth}|}
        \hline
        \begin{verbatim}
    Before implementing the reward function, please follow these steps:

    1. First, analyze the task description and environment code carefully:
        - What is the main goal?
        - What are the key state variables?
        - What are the constraints?
    
    2. Identify the key objectives and constraints:
        - Primary objectives
        - Secondary objectives
        - Physical constraints
        - Action constraints
    
    3. Break down how different actions should be rewarded:
        - What behaviors should be encouraged?
        - What behaviors should be discouraged?
        - How to balance exploration vs exploitation?
    
    4. Consider potential edge cases:
        - Boundary conditions
        - Extreme scenarios
        - Potential exploitation of the reward
    
    5. Design your reward components:
        - Base reward components
        - Shaping rewards
        - Normalization approach
    
    Format your response as follows:
    1. Start with "Reasoning:" followed by your detailed analysis of each step above
    2. Then provide your code implementation in a Python code block (```python ... ```)        
        \end{verbatim} \\
        \hline
        \end{tabular}
        \end{center}
        \end{table*}

        \begin{table*}[htbp]
            \caption{Reflection Instructions Prompt}
            \label{tab:reflection_instructions}
            \begin{center}
            \begin{tabular}{|p{\textwidth}|}
            \hline
            \begin{verbatim}
Please analyze the policy feedback and provide an improved reward function 
to better solve the task. 
Tips for analyzing feedback:
1. If a reward component's values are nearly identical, or it increases when 
it should decrease (minimize) and vice versa, consider the following options:
    a. Change its scale or temperature parameter.
    b. Re-write the component.
    c. Discard the component.
2. If a component's magnitude is significantly larger, re-scale it.
3. You want to maximize the fitness score as it is the ground truth evaluator.
4. You want to maximize positive reward components values during training and 
minimize negative reward components values.
5. If the fitness score is not improving during training, try to change the reward 
function.
Analyze each existing reward component first, then write the new reward function
code proposal.  
            \end{verbatim} \\
            \hline
            \end{tabular}
            \end{center}
            \end{table*}

This iterative process continues until the desired performance is achieved or the maximum number of iterations is reached. Each iteration builds upon the insights gained from previous rounds, incorporating feedback and performance metrics to guide the optimization process.

\section{Examples of CoT-based Reward Generation and Reflection}
This section presents concrete examples of how the Chain-of-Thought (CoT) approach is applied in practice for reward function engineering. We showcase two key aspects: the initial reward generation process and subsequent reflection-based refinement.

The first example demonstrates how the CoT approach guides the initial generation of a reward function. It shows both the analytical reasoning process and the resulting implementation for iteration 0 of the reward engineering process.

\begin{table*}[htbp]
    \caption{Analysis from llm with CoT in Iteration 0}
    \label{tab:Analysis_CoT_0}
    \begin{center}
    \begin{tabular}{|p{\textwidth}|}
    \hline
    \begin{verbatim}
        Analysis:

Breaking down the task description and environment code reveals that the primary 
objective is to make the biped navigate from the left end of the terrain to the 
right end while avoiding obstacles, falling down, or getting stuck.

Key state variables include the biped's position (x, y), velocity (vx, vy), and 
motor speeds for each joint. The constraints are:

* Physical: the biped must not fall down, get stuck, or exceed the terrain boundaries.
* Action: the agent can control the motor speeds of each joint.

Primary objectives:

* Reach the right end of the terrain
* Avoid falling down, getting stuck, or exceeding 
the terrain boundaries

Secondary objectives:

* Move forward as much as possible
* Go as fast as possible (minimize episode steps)

Reward components:

1. **Reach reward**: encourage reaching the right end of the terrain.
2. **Avoidance reward**: penalize falling down, getting stuck, or exceeding 
the terrain boundaries.
3. **Forward movement reward**: encourage moving forward.
4. **Speed reward**: incentivize fast movement.

Normalization approach:

* Use np.exp to normalize rewards and ensure they are always positive.


    \end{verbatim} \\
    \hline
    \end{tabular}
    \end{center}
    \end{table*}

    \begin{table*}[htbp]
    \caption{Code from llm with CoT in Iteration 0}
    \label{tab:Code_CoT_0}
    \begin{center}
    \begin{tabular}{|p{\textwidth}|}
    \hline
    \begin{verbatim}
def compute_reward(self, pos, action, state, terminated):
    reach_reward = 0
    avoidance_reward = 0
    forward_movement_reward = 0
    speed_reward = 0

    # Reward reaching the right end of the terrain
    if pos[0] > (TERRAIN_LENGTH - TERRAIN_GRASS) * TERRAIN_STEP:
        reach_reward = 10.0

    # Penalize falling down, getting stuck, or exceeding the terrain boundaries
    if terminated:
        avoidance_reward = -5.0

    # Encourage moving forward
    forward_movement_reward = state[2]  # vx component of velocity

    # Incentivize fast movement
    speed_reward = np.exp(state[3])  # vy component of velocity 
    # (positive reward for high speed)

    individual_reward = {
        'reach': reach_reward,
        'avoidance': avoidance_reward,
        'forward_movement': forward_movement_reward,
        'speed': speed_reward
    }

    reward = reach_reward + avoidance_reward + forward_movement_reward + speed_reward

    return reward, individual_reward
    \end{verbatim} \\
    \hline
    \end{tabular}
    \end{center}
    \end{table*}

The second example illustrates how the CoT approach facilitates reflection on reward function performance. It includes detailed performance metrics, analysis of reward components, and subsequent refinements based on observed behavior.

\begin{table*}[htbp]
    \caption{Reflection of Components in Iteration 1}
    \label{tab:Analysis_CoT_01}
    \begin{center}
    \begin{tabular}{|p{\textwidth}|}
    \hline
    \begin{verbatim}
Performance Analysis:
   reach: [0.0, 0.0, 0.0, 0.0, 0.0, 0.0, 0.0, 0.0, 0.0, 0.0].
    Max: 0.0 - Mean: 0.0 - Min: 0.0 
   forward_movement: [38.14766542628862, 100.18565675155529, 82.92425245350518, 
   225.32142357140503, 728.2149681850315, 61.379876609731, 69.62986257168465, 
   97.01515441786847, 62.27421109898911, 886.024284332291]. 
   Max: 886.024284332291 - Mean: 235.11173554183497 - Min: 38.14766542628862 
   speed: [nan, nan, 74.75, 345.75, 800.0, 44.9375, 50.125, 63.0625, 47.875, 477.75]. 
   Max: nan - Mean: nan - Min: nan 
   avoidance: [0.0, -0.1875, -0.5, -0.3125, 0.0, -0.5, -0.5, -0.5, -0.5, -0.375]. 
   Max: 0.0 - Mean: -0.3375 - Min: -0.5 
   feature: [52.5, 25.0, 17.5, 123.75, 727.5, 40.0, 40.0, 40.0, 71.25, 655.0]. 
   Max: 727.5 - Mean: 179.25 - Min: 17.5 
   fitness_score: [-9.752663325726997, -44.71357389366377, -89.69247023539725, 
   -45.75511213654681, 100.08154316943032, -94.92986408164555, -95.32390355962065,
   -91.24470863744043, -97.31017249609376, 77.49146458473439]. 
   Max: 100.08154316943032 - Mean: -39.114946061197045 - Min: -97.31017249609376 
   reward: [nan, nan, 174.67425264045596, 694.5089239552617, 2255.714967980981, 
   145.81737657636404, 159.2548627257347, 199.57765482366085, 180.89921092242002, 
   2018.3992846086621]. Max: nan - Mean: nan - Min: nan 
   episode_length: [1600.0, 1100.25, 149.5, 691.5, 1600.0, 89.875, 100.25, 126.125, 
   95.75, 955.5].
    Max: 1600.0 - Mean: 650.875 - Min: 89.875 

Best Performing Reward Function:
    # Generated code 
    def compute_reward(self, pos, action, state, terminated):
        reach_reward = 10 if np.linalg.norm(pos) < 5 else 0
        forward_movement_reward = 3 * np.abs(state[2])
        speed_reward = 0.5 * (np.exp(state[3]) - 1) / (np.max(np.exp(state[3])) - 1)
        exploration_reward = 1 if state[4] > 20 else 0
        avoidance_reward = -0.5 if terminated else 0
        feature_reward = 10 if state[5] == 1 else 0
        point_reward = 5 if state[6] == 2 else 0
        individual_reward = {
            'reach': reach_reward,
            'forward_movement': forward_movement_reward,
            'speed': speed_reward,
            'exploration': exploration_reward,
            'avoidance': avoidance_reward,
            'feature': feature_reward,
            'point': point_reward
        }
        reward = reach_reward + forward_movement_reward + speed_reward + exploration_reward 
        + avoidance_reward + feature_reward + point_reward
        return reward, individual_reward
    \end{verbatim} \\
    \hline
    \end{tabular}
    \end{center}
    \end{table*}


\begin{table*}[htbp]
    \caption{Analysis from llm with CoT in Iteration 1}
    \label{tab:Analysis_CoT_1}
    \begin{center}
    \begin{tabular}{|p{\textwidth}|}
    \hline
    \begin{verbatim}
Analysis:

**Performance Analysis**

The learning curve trend is positive, indicating that the agent is improving over time. 
However, the average fitness score of -39.114946061197045 suggests that the agent is not 
consistently achieving good results. The maximum fitness score of 100.08154316943032 
indicates that there are instances where the agent performs well, but it's unclear what 
triggers these successful behaviors.

**Reward Function Analysis**

* `reach_reward`: Encourages the agent to reach a certain position within a threshold 
(5 units). This might be effective in guiding the agent towards the goal.
* `forward_movement_reward`, `speed_reward`, and `exploration_reward`: These rewards 
are designed to encourage the agent to move forward, increase its speed, or explore 
the environment. However, these components might not provide enough motivation for 
the agent to perform complex behaviors.
* `avoidance_reward`: This reward discourages termination of the episode, but it's 
unclear how effective it is in preventing premature termination.

The reward scales seem to be relatively high, which could lead to unstable learning 
or overfitting. The `point_reward` and `feature_reward` components have a binary value 
(0 or 1), which might not provide enough nuance for the agent to learn complex behaviors.

**Behavioral Analysis**

Based on the provided information, it's unclear what strategies the agent has learned. 
However, the presence of multiple reward components suggests that the agent is learning 
to navigate the environment and interact with different features and points.
The learning trend indicates that the agent is improving over time, but there are instances 
where it performs poorly (average fitness score). This might suggest that the agent is not 
consistently learning effective strategies or is prone to overfitting.

**Improvement Planning**

1. **Reduce reward scales**: Decrease the magnitude of the rewards to avoid unstable 
learning and overfitting.
2. **Make reward components more nuanced**: Replace binary rewards with continuous 
values to encourage more complex behaviors.
3. **Improve exploration**: Increase the `exploration_reward` or introduce new 
components that promote exploration, such as a penalty for stagnation or a bonus 
for discovering new features.
4. **Tune the avoidance reward**: Adjust the `avoidance_reward` to prevent premature 
termination while still encouraging the agent to explore and interact with the environment.

    \end{verbatim} \\
    \hline
    \end{tabular}
    \end{center}
    \end{table*}

    \begin{table*}[htbp]
    \caption{Code from llm with CoT in Iteration 1}
    \label{tab:Code_CoT_1}
    \begin{center}
    \begin{tabular}{|p{\textwidth}|}
    \hline
    \begin{verbatim}
def compute_reward(self, pos, action, state, terminated):
    reach_reward = 0.5 if np.linalg.norm(pos) < 2 else 0
    forward_movement_reward = 1 * np.abs(state[2])
    speed_reward = 0.25 * (np.exp(state[3]) - 1) / (np.max(np.exp(state[3])) - 1)
    
    exploration_reward = 0.5 if state[4] > 10 else 0
    avoidance_reward = -0.25 if terminated else 0
    
    feature_reward = 2 if state[5] == 1 else 0
    point_reward = 1 if state[6] == 2 else 0
    
    individual_reward = {
        'reach': reach_reward,
        'forward_movement': forward_movement_reward,
        'speed': speed_reward,
        'exploration': exploration_reward,
        'avoidance': avoidance_reward,
        'feature': feature_reward,
        'point': point_reward
    }
    
    reward = reach_reward + forward_movement_reward + speed_reward 
    + exploration_reward + avoidance_reward + feature_reward + point_reward
    
    return reward, individual_reward
    \end{verbatim} \\
    \hline
    \end{tabular}
    \end{center}
    \end{table*}
\end{document}
